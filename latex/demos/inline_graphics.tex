\section{Inline Graphics Demo}

This template supports a variety of inline graphics, which can be used for a variety of purposes like controller inputs (ie. \controllerA+\controllerJoystickUp to attack up) or for specific dice rolls (ie. \diceSix\diceSix). They can be used by just writing \mintinline{latex}{\controllerA} (replace with whichever inline graphic you want to use) directly within text, no need to fiddle with environments or filepaths or anything like that.

A complete list of all possible inline graphics are included:

\subsection{Controls}

\newcommand{\glyphdemo}[1]{#1 \hspace{1pt} \mintinline{latex}{#1}}

\begin{itemize}
	\item \glyphdemo{\controllerA}
	\item \glyphdemo{\controllerB}
	\item \glyphdemo{\controllerX}
	\item \glyphdemo{\controllerY}
	\item \glyphdemo{\controllerDpad}
	\item \glyphdemo{\controllerDpadUp}
	\item \glyphdemo{\controllerDpadLeft}
	\item \glyphdemo{\controllerDpadDown}
	\item \glyphdemo{\controllerDpadRight}
	\item \glyphdemo{\controllerJoystick}
	\item \glyphdemo{\controllerJoystickUp}
	\item \glyphdemo{\controllerJoystickLeft}
	\item \glyphdemo{\controllerJoystickDown}
	\item \glyphdemo{\controllerJoystickRight}
	\item \glyphdemo{\controllerJoystickPress}
	\item \glyphdemo{\controllerRJoystick}
	\item \glyphdemo{\controllerRJoystickUp}
	\item \glyphdemo{\controllerRJoystickLeft}
	\item \glyphdemo{\controllerRJoystickDown}
	\item \glyphdemo{\controllerRJoystickRight}
	\item \glyphdemo{\controllerRJoystickPress}
	\item \glyphdemo{\controllerLJoystick}
	\item \glyphdemo{\controllerLJoystickUp}
	\item \glyphdemo{\controllerLJoystickLeft}
	\item \glyphdemo{\controllerLJoystickDown}
	\item \glyphdemo{\controllerLJoystickRight}
	\item \glyphdemo{\controllerLJoystickPress}
	\item \glyphdemo{\controllerL}
	\item \glyphdemo{\controllerLalt}
	\item \glyphdemo{\controllerR}
	\item \glyphdemo{\controllerRalt}
	\item \glyphdemo{\controllerZ}
	\item \glyphdemo{\controllerMouse}
	\item \glyphdemo{\controllerMouseLMB}
	\item \glyphdemo{\controllerMouseRMB}
	\item \glyphdemo{\controllerMouseScroll}
	\item \glyphdemo{\controllerMouseScrollUp}
	\item \glyphdemo{\controllerMouseScrollDown}
\end{itemize}

\subsection{Dice}

\begin{itemize}
	\item \glyphdemo{\diceOne}
	\item \glyphdemo{\diceTwo}
	\item \glyphdemo{\diceThree}
	\item \glyphdemo{\diceFour}
	\item \glyphdemo{\diceFive}
	\item \glyphdemo{\diceSix}
\end{itemize}

\newpage
