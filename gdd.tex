% importing packages and setting up the document
\documentclass[12pt]{article}

% set margins on page
\usepackage{geometry}
\geometry{a4paper,
		  lmargin=2.5cm,
		  rmargin=2.5cm,
		  tmargin=2.5cm,
		  bmargin=3.0cm}

% paragraph formatting
\setlength{\parskip}{0.75em}
\setlength{\parindent}{0em}

% remove margin on lists
\usepackage{enumitem}
\setlist{nolistsep}

% improve text rendering
\usepackage[T1]{fontenc}
\usepackage{lmodern}
\usepackage{mathtools}

% support links
\usepackage[hidelinks]{hyperref}

% image support
\usepackage{pdfpages}
\usepackage{graphicx}

% text color support
\usepackage{xcolor}

% additional input options
\usepackage{menukeys}

% code
\usepackage{minted}


% commands for controller button icons in text
\newcommand*{\controllerImg}[1]{%
    \raisebox{-.17\baselineskip}{%
        \includegraphics[
        height=\baselineskip,
        width=\baselineskip,
        keepaspectratio,
        ]{#1}%
    }%
}

\newcommand{\controllerA}{\controllerImg{latex/images/controller_a.png}}
\newcommand{\controllerB}{\controllerImg{latex/images/controller_b.png}}
\newcommand{\controllerX}{\controllerImg{latex/images/controller_x.png}}
\newcommand{\controllerY}{\controllerImg{latex/images/controller_y.png}}
\newcommand{\controllerDpad}{\controllerImg{latex/images/controller_dpad.png}}
\newcommand{\controllerDpadUp}{\controllerImg{latex/images/controller_dpad_up.png}}
\newcommand{\controllerDpadLeft}{\controllerImg{latex/images/controller_dpad_left.png}}
\newcommand{\controllerDpadDown}{\controllerImg{latex/images/controller_dpad_down.png}}
\newcommand{\controllerDpadRight}{\controllerImg{latex/images/controller_dpad_right.png}}
\newcommand{\controllerJoystick}{\controllerImg{latex/images/controller_joystick.png}}
\newcommand{\controllerJoystickUp}{\controllerImg{latex/images/controller_joystick_up.png}}
\newcommand{\controllerJoystickLeft}{\controllerImg{latex/images/controller_joystick_left.png}}
\newcommand{\controllerJoystickDown}{\controllerImg{latex/images/controller_joystick_down.png}}
\newcommand{\controllerJoystickRight}{\controllerImg{latex/images/controller_joystick_right.png}}
\newcommand{\controllerL}{\controllerImg{latex/images/controller_l.png}}
\newcommand{\controllerLalt}{\controllerImg{latex/images/controller_l2.png}}
\newcommand{\controllerR}{\controllerImg{latex/images/controller_r.png}}
\newcommand{\controllerRalt}{\controllerImg{latex/images/controller_r2.png}}


% commands for specific page formatting
\newcommand{\gddTitlepage}{
\newpage

\vspace*{3cm}

\begin{center}
{\Huge \gddTitle}

{\small \gddSubtitle}
\end{center}

\vspace{5cm}

\begin{center}
{\Large \gddCompany}

{\normalsize Authored by: \gddAuthors}
\end{center}

\vspace{7cm}

\begin{center}
{\normalsize Date: \gddDate}

{\normalsize Version: \gddVersion}
\end{center}
\newpage
}

\newcommand{\gddChangelogEntry}[3]{
\begin{minipage}[t]{0.2\textwidth}
\begin{raggedright}
\textbf{v#1}
\newline
{\small #2}
\end{raggedright}
\end{minipage}
\hfill
\begin{minipage}[t]{0.785\textwidth}
#3
\end{minipage}

\hrulefill
\newline
}

\newcommand{\gddChangelog}{
\newpage

\gddChangelogEntries

\newpage
}


\newcommand{\gddTitle}{Game Name}
\newcommand{\gddSubtitle}{Game Subtitle}
\newcommand{\gddCompany}{Company Name}
\newcommand{\gddAuthors}{Firstname Lastname, Firstname Lastname, Firstname Lastname}
\newcommand{\gddDate}{January 1st, 1960}
\newcommand{\gddVersion}{1.0.0}

\newcommand{\gddChangelogEntries}{
\gddChangelogEntry{0.1.0}{January 1st, 1960}{
Changelog List\begin{itemize}
\item Point A
\item Point B
\item Point C
\end{itemize}}
\gddChangelogEntry{0.0.1}{January 1st, 1960}{Changelog Modified}
\gddChangelogEntry{0.0.0}{January 1st, 1960}{Changelog Created}
}

\begin{document}

\gddTitlepage
\gddChangelog

\section{Concept}
\subsection{Genre}
What high-level genre this game belongs to, or what genres it most closely aligns to if it can't be easily classified as one.
\subsection{Philosophy}
What the guiding philosophy of the game's design is.
\subsection{Goals}
What the core goals of the game sets out to accomplish.
\subsection{Target Audience}
Who is the main audience(s) the game is intended to be for, and what is done to appeal to them.
\subsection{Selling Points}
What the selling points of the game is.
\subsection{Inspirations}
What sources of inspiration there are, or what works can be used for reference.
\subsection{Scope}
How large the project is, and what significant sources of work are there.

\section{Abstract Design}
\subsection{Mood}
What the general mood is that the game tries to convey.
\subsection{Art Direction}
What the total art direction of the game is.
\subsection{Inclusivity}
\subsubsection{Racial}
What is done to be inclusive of various racial identities, primarily (but not limited to) skin color.
\subsubsection{Cultural}
What is done to accommodate various different cultures.
\subsubsection{Gender}
What is done to be inclusive of people of every gender (or lack thereof). Groups to think about include (but are not limited to) men and women, cisgender and transgender people, nonbinary people, and gender non-conforming people.
\subsubsection{Physical Disabilities}
What is done to accommodate for people with physical disabilities.
\subsubsection{Sensory Conditions}
What is done to accommodate for people with sensory conditions. This includes colorblindness, lack of vision, hard of hearing or lack of hearing.

\section{Mechanical Design}
\subsection{Gameplay Overview}
What the (potentially multiple forms of) gameplay of the game looks like, in general.
\subsection{Game Mechanics}
What the finer game mechanics and game rules are.
\subsection{Controls}
What the controls for the game are, and what different options for input there is.

You can create keyboard input visuals with \keys{\ctrl + \shift + p} or controller buttons like \controllerA.

Codes for various controller buttons can be found as:
\begin{itemize}
\item \controllerA \hspace{1pt} \mintinline{latex}{\controllerA}
\item \controllerB \hspace{1pt} \mintinline{latex}{\controllerB}
\item \controllerX \hspace{1pt} \mintinline{latex}{\controllerX}
\item \controllerY \hspace{1pt} \mintinline{latex}{\controllerY}
\item \controllerDpad \hspace{1pt} \mintinline{latex}{\controllerDpad}
\item \controllerDpadUp \hspace{1pt} \mintinline{latex}{\controllerDpadUp}
\item \controllerDpadLeft \hspace{1pt} \mintinline{latex}{\controllerDpadLeft}
\item \controllerDpadDown \hspace{1pt} \mintinline{latex}{\controllerDpadDown}
\item \controllerDpadRight \hspace{1pt} \mintinline{latex}{\controllerDpadRight}
\item \controllerJoystick \hspace{1pt} \mintinline{latex}{\controllerJoystick}
\item \controllerJoystickUp \hspace{1pt} \mintinline{latex}{\controllerJoystickUp}
\item \controllerJoystickLeft \hspace{1pt} \mintinline{latex}{\controllerJoystickLeft}
\item \controllerJoystickDown \hspace{1pt} \mintinline{latex}{\controllerJoystickDown}
\item \controllerJoystickRight \hspace{1pt} \mintinline{latex}{\controllerJoystickRight}
\item \controllerL \hspace{1pt} \mintinline{latex}{\controllerL}
\item \controllerLalt \hspace{1pt} \mintinline{latex}{\controllerLalt}
\item \controllerR \hspace{1pt} \mintinline{latex}{\controllerR}
\item \controllerRalt \hspace{1pt} \mintinline{latex}{\controllerRalt}
\end{itemize}

\section{UI Design}
\subsection{Theme}
What is the overall theme of the UI and what is the direction it takes.
\subsection{Menus}
What do various menus, such as the main menu or a pause menu look like.
\subsection{HUD}
What does the ingame HUD look like.

\section{World Design}
\subsection{Environment}
What the world itself looks like, from a broad view.
\subsection{Objects}
What individual objects, or classes of objects, populate the world.
\subsection{Lore}
What existing background to the world exists, that may or may not be explored within the game itself.
\subsection{Characters}
What characters exist in the game.
\subsection{Story}
What story does the game follow.

\section{Technical Design}
\subsection{Engine}
What existing or custom engine is the game to use, and what features are required of it.
\subsection{Target Platforms}
Which platforms or consoles is the game intended to play on.

\section{Business}
\subsection{Monetization}
How the game is intended to make money.
\subsection{Schedule}
What sort of timeline development is expected to take.

\end{document}